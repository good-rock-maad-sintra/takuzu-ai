\documentclass[12pt,a4paper]{article}
\usepackage[legalpaper, portrait, margin=2cm]{geometry}
\usepackage{fancyhdr}
\usepackage{amsmath}
\usepackage{amssymb}
\usepackage{graphicx}
\usepackage{wrapfig}
\usepackage{blindtext}
\usepackage{hyperref}
\usepackage{enumitem}
\usepackage{pdflscape}
\usepackage{svg}
\usepackage{listings}
\usepackage{xcolor}
\usepackage{adjustbox}
\usepackage{booktabs}
\usepackage{float}

\graphicspath{ {./} }
\hypersetup{
  colorlinks=true,
  linkcolor=blue,
  filecolor=magenta,
  urlcolor=blue,
  citecolor=blue,
  pdftitle={Relatório Projeto IA - 2021/2022},
  pdfpagemode=FullScreen,
}

\pagestyle{fancy}
\fancyhf{}
\rhead{Grupo \textbf{5}}
\lhead{Relatório Projeto IA (Takuzu) 2021/2022 LEIC-A}
\cfoot{Diogo Gaspar (99207) e João Rocha (99256)}

\renewcommand{\footrulewidth}{0.2pt}

\renewcommand{\labelitemii}{$\circ$}
\renewcommand{\labelitemiii}{$\diamond$}
\newcommand{\op}{\text}

\newlist{constraintsList}{itemize}{4}
\setlist[constraintsList]{itemsep=1pt, topsep=1pt, label=\protect\mpbullet}

\begin{document}

\section*{Descrição do Problema e da Solução}

Foi proposta a elaboração de um programa, em \texttt{Python}, que resolvesse, de
forma eficiente, o \textit{puzzle} binário \textit{takuzu}.
Este \textit{puzzle} pede-nos para encontrar um tabuleiro, totalmente preenchido
com 0's e 1's, partindo de uma configuração inicial, que satisfaça as seguintes restrições:
\begin{itemize}
      \item Não podem haver 3 símbolos (0's ou 1's) iguais consecutivos;
      \item A diferença entre o número de 0's e 1's numa dada linha ou coluna deve
            ser no máximo 1: 0 em tabuleiros de tamanho par, 1 nos de tamanho ímpar;
      \item Todas as linhas devem ser diferentes entre si;
      \item Todas as colunas devem ser diferentes entre si.
\end{itemize}

No contexto da matéria de Inteligência Artificial, isto consiste em encontrar um
estado \textbf{completo} e \textbf{consistente}, onde:
\begin{itemize}
      \item Um estado diz-se \textbf{completo} se não tiver células vazias;
      \item Um estado diz-se \textbf{consistente} se satisfizer as 4 regras mencionadas acima.
\end{itemize}

Chegar a um estado completo é bastante fácil, bastando para tal preencher todas as células do tabuleiro.
Para garantir que atingimos um estado completo \textbf{e} consistente, basta então
garantir que nunca fazemos uma jogada que nos leve a um estado inconsistente.

% NOTE: se tivermos espaço, mantemos isto, caso contrário apagamos, não é assim tão relevante

\begin{figure}[H]
      \centering
      \includesvg[width=0.1\textwidth]{takuzu-initial-state}
      \hspace*{1cm}
      \includesvg[width=0.1\textwidth]{takuzu-final-state}
      \caption{Estado inicial e final de um possível \textit{puzzle} takuzu consistente}
      \label{fig:takuzu-initial-state}
\end{figure}

Para nos ajudar a resolver o problema em mãos, vamos suportar-nos em algumas definições auxiliares.
Dizemos que uma jogada é \textbf{impossível} se a execução desta levar a um estado
inconsistente (o estado quebra alguma das regras supra-mencionadas, portanto).
Destas regras, a única cuja verificação merece alguma explicação é a da identificação
linhas/colunas repetidas.
Para garantir que tal nunca acontece, guardamos (em \texttt{Board}) a qualquer momento
dois \textit{sets}, cada um guardando \textit{strings} binárias que representam,
respetivamente, as linhas e colunas que já estão totalmente preenchidas no tabuleiro.
Esta solução com \textit{strings} binárias permite aumentar consideravelmente a
\textbf{eficiência} da verificação de igualdade entre linhas/colunas: é mais eficiente
comparar \textit{strings} que tuplos, por exemplo.

As jogadas podem ainda ser \textbf{possíveis} (corresponde apenas a não ser impossível)
ou \textbf{obrigatórias} (se forem possíveis e o seu conjugado não for).
Aqui, o conjugado de uma jogada corresponde à jogada que atua sobre a mesma célula,
mas com o valor \textbf{conjugado}: isto é, se uma jogada coloca 0 na posição (1,3),
a sua conjugada coloca 1 nessa mesma posição.

Em cada estado, verificamos sempre se há alguma célula vazia em que ambas as jogadas
sejam impossíveis.
Neste caso, qualquer jogada nessa célula levaria a um estado em que o tabuleiro é
inconsistente e não vale a pena prosseguir neste ramo da árvore de procura.

Sempre que haja ações obrigatórias por realizar, num dado estado, realizamo-las.
Isto é apenas lógico, visto que, por definição de obrigatoriedade, vamos ter de
as executar para chegar a qualquer tabuleiro solução (considerando o ramo
atual da árvore de procura, claro: uma jogada obrigatória num dado ramo poderá
não o ser na solução).
Assim, antecipando a sua execução, \textbf{reduzimos o número de nós da nossa árvore de procura}.

Sempre que tal não é possível, escolhemos um par de ações possíveis para qualquer
célula vazia do tabuleiro (note-se que uma vez que não há jogadas obrigatórias -
células em que apenas uma jogada é possível - nem células em que não seja possível jogar,
é necessariamente verdade que em qualquer célula vazia podemos colocar tanto um 0 como um 1).

Esta decisão de devolver sempre no máximo duas ações traduz-se em que o \textbf{\textit{branching factor}}
da nossa árvore seja 2.
Como vamos ver na análise experimental, isto é fundamental para a execução em tempo
eficiente da nossa solução.

Como a nossa procura é feita de forma a nunca alcançar estados inconsistentes, basta
que o nosso \texttt{goal\_test} verifique se está num estado completo - um estado
sem células vazias (nesse caso será necessariamente uma solução).

Tendo em conta a perspetiva dos CSP's (\textit{Constraint Satisfaction Problems})
abordada em aula, temos que na nossa solução:
\begin{itemize}
      \item A opção de devolver \textbf{no máximo duas opções}, ambas respetivas a apenas uma
            posição vazia, capitaliza na ideia de escolher uma variável de cada vez, visto
            que todas vão ter de ser escolhidas eventualmente.
            Como vimos em aula, isto pode reduzir o número de nós da árvore de procura
            em várias ordens de grandeza.
      \item A opção de devolver as jogadas obrigatórias sempre que possível é uma
            aplicação da heurística LCV (\textit{Least Constraining Value}).
            De facto, ao escolhermos um valor obrigatório para a variável não estamos
            a impor qualquer condição ao tabuleiro que ainda não estivesse imposta
            (mesmo que indiretamente).
\end{itemize}

\section*{Função Heurística}

Na nossa escolha de função heurística, insistimos na ideia do LCV.
Como a procura A* escolhe nós por ordem crescentes da função $f(n) = g(n) + h(n)$, aos valores mais constringentes devem estar atribuídos valores da heurística maiores.
Para isto, calculamos um "peso" que corresponde à média de duas componentes:
\begin{itemize}
      \item O constrangimento ao longo da linha/coluna da posição onde foi executada a última jogada, causado por essa jogada.
            Idealizamos que uma jogada é tão mais constringente sobre uma linha quão mais perto deixar essa linha de estar saturada do valor que acabou de ser introduzido.
            Idealizamos que uma jogada é tão mais constringente sobre uma linha quão mais perto deixar essa linha de estar saturada do valor que acabou de ser introduzido.
            Idealizamos que uma jogada é tão mais constringente sobre uma linha quão mais perto deixar essa linha de estar saturada do valor que acabou de ser introduzido.
            Idealizamos que uma jogada é tão mais constringente sobre uma linha quão mais perto deixar essa linha de estar saturada do valor que acabou de ser introduzido.
            Idealizamos que uma jogada é tão mais constringente sobre uma linha quão mais perto deixar essa linha de estar saturada do valor que acabou de ser introduzido.
            Assim, calculamos o constrangimento sobre uma linha tirando a proporção de valores preenchidos que estão preenchidos com o valor que acabou de ser inserido.
            O valor final desta componente é a média entre o constrangimento sobre a linha e o constrangimento sobre a coluna.
      \item O constrangimento na "vizinhança" da posição onde foi executada a última jogada, causado por essa jogada.
            Uma jogada é tão mais constringente sobre a sua vizinhança quantas mais jogadas se tornarem obrigatórias para prevenir 3 símbolos idênticos consecutivos.
            Esta componente calcula então o número média de tais jogadas originadas.
\end{itemize}

O valor final da heurística é então o peso - que corresponde à média das duas componentes acima - multiplicado pelo número de células vazias.
Ao multiplicarmos pelo número de células vazias fazemos com que a heurística seja \textbf{mais dominante}.
De facto, se usássemos apenas o peso, a nossa procura A* não seria muito diferente da DFS.

Observe-se que, como o valor do peso está entre 0 e 1, a nossa heurística é \textbf{admissível}, visto que a distância verdadeira ao objetivo é o número de células vazias, se estivermos num ramo que leva ao objetivo, e infinito caso contrário.
Note-se no entanto que a heurística não é \textbf{consistente}.
Desta forma, e uma vez que a função \texttt{astar\_search} implementa uma procura em grafo, a nossa heurística não garante uma procura \textbf{ótima}.
Contudo, uma vez que todas as soluções estão à mesma distância do estado inicial, isto não é relevante.

Na verdade, este facto torna as procuras informadas pouco mais úteis que as procuras cegas.
A procura informada acaba por servir apenas para fazer escolhas mais "educada".
Porém, como observamos na análise experimental, isto nem sequer leva a melhores resultados.

\section*{Análise Experimental}

Tendo em conta a implementação descrita na página anterior, foram obtidos
os resultados experimentais (para os testes públicos fornecidos pela docência)
descritos na \textbf{Tabela 1}. Note-se que a coluna \texttt{Tempo de Execução (ms)}
corresponde à media de tempo de execução de cada teste, calculada recorrendo à ferramenta
\texttt{hyperfine} (com 250 execuções por teste, por procura). A heurística
utilizada para as procuras A* e Gananciosa é a referida na secção anterior.

Considerámos, aquando da implementação da solução, que escolher \textbf{uma variável por nível}
ajudaria de forma drástica quanto à eficiência do problema. Ao analisar os resultados
obtidos (encontram-se na \textbf{Tabela 2}) com esta variação da solução, ficámos
relativamente surpreendidos com o facto das diferenças terem sido mínimas em comparação
com a implementação original: o único caso em que a diferença é de facto notória
é no \textbf{sexto teste público}, onde o tempo de execução varia entre cerca de
900ms, na DFS, para tempos enormes, nas outras procuras.

O teste em questão não
aparenta ter pormenores significativamente extravagantes, em comparação com os
outros tabuleiros de teste, exceto a diferença entre o \textbf{número de nós gerados e expandidos},
que só acontece neste teste e em dois outros (testes esses que não
veem o seu tempo de execução aumentar a este nível).
Estamos em crer que a nossa implementação só ficaria verdadeiramente
posta à prova com tabuleiros de teste muito grandes e bastante vazios, por forma
a evidenciar a eficiência da solução que escolhe uma variável por nível.

A heurística \textbf{LCV} surtiu, contrastando com a utilização de 1 variável por nível,
uma diferença drástica em comparação com a sua não-utilização: apenas os
dois primeiros testes corriam em tempo útil, para qualquer procura, considerando
também a ausência da lógica de 1 variável por nível. Estando essa lógica
incluída, os resultados experimentais são os descritos na \textbf{Tabela 3}.

\section*{Conclusão}

A heurística obtida na segunda secção, apesar de fazer (na nossa opinião) bastante
sentido quando aplicada ao problema em questão, não surte efeitos particularmente
positivos na execução da solução, quando comparada com a procura em profundidade
primeiro - possivelmente em tabuleiros diferentes tal ocorreria.

Na solução submetida para avaliação automática, via \textit{Mooshak}, optámos
por utilizar a versão original, com 1 variável por nível, utilizando uma
procura em profundidade primeiro (que, segundo os resultados experimentais obtidos,
apresentava o melhor desempenho, ainda que a diferença para a BFS seja
praticamente negligenciável).

% TODO: falar de completude e consistência e cenas para cada procura

\section*{Anexos}

% TODO: adicionar uma coluna com o tamanho do input

\begin{adjustbox}{width={\textwidth}, totalheight={\textheight}, keepaspectratio}
      \begin{tabular}{l cccc cccc cccc}
            \toprule
                  & \multicolumn{4}{c}{Tempo de Execução (ms)} & \multicolumn{4}{c}{Nós Gerados} & \multicolumn{4}{c}{Nós Expandidos}                                                                            \\
            \cmidrule(lr){2-5} \cmidrule(lr){6-9} \cmidrule(lr){10-13}
            Teste & BFS                                        & DFS                             & A*                                 & Gananciosa & BFS & DFS & A*  & Gananciosa & BFS & DFS & A*  & Gananciosa \\
            \midrule
            01    & 73.128                                     & 72.515                          & 73.633                             & 73.186     & 7   & 7   & 7   & 7          & 7   & 7   & 7   & 7          \\
            02    & 72.843                                     & 73.416                          & 73.337                             & 73.053     & 7   & 7   & 7   & 7          & 7   & 7   & 7   & 7          \\
            03    & 81.326                                     & 81.748                          & 82.538                             & 82.823     & 43  & 43  & 43  & 43         & 43  & 42  & 43  & 43         \\
            04    & 77.715                                     & 75.694                          & 77.806                             & 77.450     & 32  & 32  & 32  & 32         & 32  & 32  & 32  & 32         \\
            05    & 85.999                                     & 85.665                          & 86.816                             & 86.653     & 59  & 59  & 59  & 59         & 59  & 58  & 59  & 59         \\
            06    & 111.490                                    & 110.929                         & 113.014                            & 112.789    & 85  & 82  & 85  & 85         & 85  & 81  & 85  & 85         \\
            07    & 91.672                                     & 91.307                          & 92.653                             & 92.878     & 69  & 69  & 69  & 69         & 69  & 69  & 69  & 69         \\
            08    & 74.727                                     & 74.672                          & 75.366                             & 75.339     & 19  & 19  & 19  & 19         & 19  & 19  & 19  & 19         \\
            09    & 116.292                                    & 115.688                         & 118.930                            & 118.580    & 139 & 139 & 139 & 139        & 139 & 139 & 139 & 139        \\
            10    & 154.169                                    & 151.870                         & 156.289                            & 156.253    & 184 & 184 & 184 & 184        & 184 & 184 & 184 & 184        \\
            11    & 129.147                                    & 129.426                         & 133.098                            & 133.109    & 180 & 180 & 180 & 180        & 180 & 180 & 180 & 180        \\
            12    & 98.991                                     & 99.399                          & 102.951                            & 102.772    & 166 & 166 & 166 & 166        & 166 & 166 & 166 & 166        \\
            13    & 102.463                                    & 102.535                         & 107.096                            & 107.023    & 180 & 180 & 180 & 180        & 180 & 180 & 180 & 180        \\
            \bottomrule
      \end{tabular}
\end{adjustbox}

\vspace*{0.5cm}

\begin{center}
      Tabela 1: Resultados Experimentais, 1 variável por nível.
\end{center}

\vspace*{1cm}

\begin{adjustbox}{width={\textwidth}, totalheight={\textheight}, keepaspectratio}
      \begin{tabular}{l cccc cccc cccc}
            \toprule
                  & \multicolumn{4}{c}{Tempo de Execução (ms)} & \multicolumn{4}{c}{Nós Gerados} & \multicolumn{4}{c}{Nós Expandidos}                                                                              \\
            \cmidrule(lr){2-5} \cmidrule(lr){6-9} \cmidrule(lr){10-13}
            Teste & BFS                                        & DFS                             & A*                                 & Gananciosa & BFS & DFS  & A*  & Gananciosa & BFS & DFS  & A*  & Gananciosa \\
            \midrule
            01    & 72.970                                     & 73.094                          & 73.127                             & 73.297     & 7   & 7    & 7   & 7          & 7   & 7    & 7   & 7          \\
            02    & 73.474                                     & 73.014                          & 73.396                             & 73.446     & 7   & 7    & 7   & 7          & 7   & 7    & 7   & 7          \\
            03    & 82.725                                     & 81.771                          & 82.583                             & 82.932     & 60  & 49   & 51  & 51         & 57  & 42   & 44  & 44         \\
            04    & 76.408                                     & 76.588                          & 77.628                             & 77.514     & 32  & 32   & 32  & 32         & 32  & 32   & 32  & 32         \\
            05    & 95.799                                     & 86.219                          & 93.721                             & 92.753     & 209 & 80   & 149 & 140        & 189 & 57   & 114 & 104        \\
            06    & -                                          & 905.868                         & -                                  & -          & -   & 9537 & -   & -          & -   & 9430 & -   & -          \\
            07    & 92.043                                     & 92.060                          & 93.145                             & 92.655     & 69  & 69   & 69  & 69         & 69  & 69   & 69  & 69         \\
            08    & 74.632                                     & 74.724                          & 74.974                             & 75.032     & 19  & 19   & 19  & 19         & 19  & 19   & 19  & 19         \\
            09    & 115.589                                    & 116.284                         & 118.702                            & 118.603    & 139 & 139  & 139 & 139        & 139 & 139  & 139 & 139        \\
            10    & 152.918                                    & 151.920                         & 156.084                            & 156.218    & 184 & 184  & 184 & 184        & 184 & 184  & 184 & 184        \\
            11    & 128.591                                    & 129.511                         & 133.295                            & 133.087    & 180 & 180  & 180 & 180        & 180 & 180  & 180 & 180        \\
            12    & 98.737                                     & 99.028                          & 102.885                            & 102.364    & 166 & 166  & 166 & 166        & 166 & 166  & 166 & 166        \\
            13    & 102.898                                    & 102.658                         & 106.931                            & 107.385    & 180 & 180  & 180 & 180        & 180 & 180  & 180 & 180        \\
            \bottomrule
      \end{tabular}
\end{adjustbox}

\vspace*{0.5cm}

\begin{center}
      Tabela 2: Resultados Experimentais, todas as ações possíveis por nível.
\end{center}

\vspace*{1cm}

\begin{adjustbox}{width={\textwidth}, totalheight={\textheight}, keepaspectratio}
      \begin{tabular}{l cccc cccc cccc}
            \toprule
                  & \multicolumn{4}{c}{Tempo de Execução (ms)} & \multicolumn{4}{c}{Nós Gerados} & \multicolumn{4}{c}{Nós Expandidos}                                                                                  \\
            \cmidrule(lr){2-5} \cmidrule(lr){6-9} \cmidrule(lr){10-13}
            Teste & BFS                                        & DFS                             & A*                                 & Gananciosa & BFS & DFS  & A*    & Gananciosa & BFS & DFS  & A*    & Gananciosa \\
            \midrule
            01    & 76.747                                     & 75.894                          & 75.294                             & 73.152     & 132 & 14   & 54    & 14         & 85  & 7    & 40    & 7          \\
            02    & 74.491                                     & 74.594                          & 74.677                             & 76.325     & 42  & 14   & 30    & 14         & 27  & 9    & 17    & 7          \\
            03    & -                                          & 88.132                          & 9088.461                           & 148.721    & -   & 108  & 13750 & 464        & -   & 76   & 9726  & 320        \\
            04    & -                                          & 82.285                          & 32412.935                          & 91.285     & -   & 64   & 23928 & 134        & -   & 43   & 14157 & 83         \\
            05    & -                                          & 93.511                          & -                                  & 143.983    & -   & 110  & -     & 412        & -   & 66   & -     & 258        \\
            06    & -                                          & 130.922                         & -                                  & -          & -   & 242  & -     & -          & -   & 179  & -     & -          \\
            07    & -                                          & 604.342                         & -                                  & 4702.732   & -   & 1800 & -     & 8874       & -   & 1758 & -     & 6717       \\
            08    & -                                          & 81.294                          & -                                  & 84.485     & -   & 38   & -     & 80         & -   & 21   & -     & 48         \\
            09    & -                                          & -                               & -                                  & -          & -   & -    & -     & -          & -   & -    & -     & -          \\
            10    & -                                          & -                               & -                                  & -          & -   & -    & -     & -          & -   & -    & -     & -          \\
            11    & -                                          & -                               & -                                  & -          & -   & -    & -     & -          & -   & -    & -     & -          \\
            12    & -                                          & -                               & -                                  & -          & -   & -    & -     & -          & -   & -    & -     & -          \\
            13    & -                                          & -                               & -                                  & -          & -   & -    & -     & -          & -   & -    & -     & -          \\
            \bottomrule
      \end{tabular}
\end{adjustbox}

\vspace*{0.5cm}

\begin{center}
      Tabela 3: Resultados Experimentais, 1 variável por nível, sem LCV.
\end{center}

\vspace*{1cm}

\textbf{Nota}: Todas as entradas identificadas com \texttt{-} nas tabelas acima correspondem a
cenários onde a procura não conseguiu resolver o teste proposto em tempo útil (cada
teste foi deixado a correr durante 2 minutos).

\end{document}
