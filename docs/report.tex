\documentclass[12pt,a4paper]{article}
\usepackage[legalpaper, portrait, margin=2cm]{geometry}
\usepackage{fancyhdr}
\usepackage{amsmath}
\usepackage{amssymb}
\usepackage{graphicx}
\usepackage{wrapfig}
\usepackage{blindtext}
\usepackage{hyperref}
\usepackage{enumitem}
\usepackage{pdflscape}
\usepackage{svg}
\usepackage{listings}
\usepackage{xcolor}
\usepackage{minted}

\graphicspath{ {./} }
\hypersetup{
  colorlinks=true,
  linkcolor=blue,
  filecolor=magenta,
  urlcolor=blue,
  citecolor=blue,
  pdftitle={Relatório Projeto IA - 2021/2022},
  pdfpagemode=FullScreen,
}

\pagestyle{fancy}
\fancyhf{}
\rhead{Grupo \textbf{5}}
\lhead{Relatório Projeto IA (Takuzu) 2021/2022 LEIC-A}
\cfoot{Diogo Gaspar (99207) e João Rocha (99256)}

\renewcommand{\footrulewidth}{0.2pt}

\renewcommand{\labelitemii}{$\circ$}
\renewcommand{\labelitemiii}{$\diamond$}
\newcommand{\op}{\text}

\newlist{constraintsList}{itemize}{4}
\setlist[constraintsList]{itemsep=1pt, topsep=1pt, label=\protect\mpbullet}

\begin{document}
  \begin{titlepage}
    \begin{center}
      \vspace*{5cm}

      \Huge
      \textbf{Projeto IA - Takuzu}

      \vspace{0.5cm}

      \LARGE
      Diogo Gaspar (99207), João Rocha (99256)

      \vspace{0.5cm}
      \Large
      Grupo 005 LEIC-A

      \vfill
    \end{center}
  \end{titlepage}

  \section*{Descrição do Problema e da Solução}

  Foi proposta a elaboração de um programa, em \texttt{Python}, que resolvesse,
  de forma eficiente, \textit{puzzles} binários.

  \section*{Análise Experimental}
  
\end{document}
