\documentclass[12pt,a4paper]{article}
\usepackage[legalpaper, portrait, margin=2cm]{geometry}
\usepackage{fancyhdr}
\usepackage{amsmath}
\usepackage{amssymb}
\usepackage{graphicx}
\usepackage{wrapfig}
\usepackage{blindtext}
\usepackage{hyperref}
\usepackage{enumitem}
\usepackage{pdflscape}
\usepackage{svg}
\usepackage{listings}
\usepackage{xcolor}
\usepackage{adjustbox}
\usepackage{booktabs}

\graphicspath{ {./} }
\hypersetup{
  colorlinks=true,
  linkcolor=blue,
  filecolor=magenta,
  urlcolor=blue,
  citecolor=blue,
  pdftitle={Relatório Projeto IA - 2021/2022},
  pdfpagemode=FullScreen,
}

\pagestyle{fancy}
\fancyhf{}
\rhead{Grupo \textbf{5}}
\lhead{Relatório Projeto IA (Takuzu) 2021/2022 LEIC-A}
\cfoot{Diogo Gaspar (99207) e João Rocha (99256)}

\renewcommand{\footrulewidth}{0.2pt}

\renewcommand{\labelitemii}{$\circ$}
\renewcommand{\labelitemiii}{$\diamond$}
\newcommand{\op}{\text}

\newlist{constraintsList}{itemize}{4}
\setlist[constraintsList]{itemsep=1pt, topsep=1pt, label=\protect\mpbullet}

\begin{document}

\section*{Descrição do Problema e da Solução}

Foi proposta a elaboração de um programa, em \texttt{Python}, que resolvesse, de
forma eficiente, o \textit{puzzle} binário \textit{takuzu}.
Este \textit{puzzle} pede-nos para encontrar um tabuleiro, totalmente preenchido
com 0's e 1's, partindo de uma configuração inicial, que satisfaça as seguintes restrições:
\begin{itemize}
      \item Não podem haver 3 símbolos (0's ou 1's) iguais consecutivos;
      \item A diferença entre o número de 0's e 1's numa dada linha ou coluna deve
            ser no máximo 1: 0 em tabuleiros de tamanho par, 1 nos de tamanho ímpar;
      \item Todas as linhas devem ser diferentes entre si;
      \item Todas as colunas devem ser diferentes entre si.
\end{itemize}

No contexto da matéria de Inteligência Artificial, isto consiste em encontrar um
estado \textbf{completo} e \textbf{consistente}, onde:
\begin{itemize}
      \item Um estado diz-se \textbf{completo} se não tiver células vazias;
      \item Um estado diz-se \textbf{consistente} se satisfizer as 4 regras mencionadas acima.
\end{itemize}

Chegar a um estado completo é bastante fácil, bastando para tal preencher todas as células do tabuleiro.
Para garantir que atingimos um estado completo \textbf{e} consistente, basta então
garantir que nunca fazemos uma jogada que nos leve a um estado inconsistente.

% TODO: se tivermos espaço, adicionar aqui um exemplo de um tabuleiro completamente preenchido?

Para nos ajudar a resolver o problema em mãos, vamos suportar-nos em algumas definições auxiliares.
Dizemos que uma jogada é \textbf{impossível} se a execução desta levar a um estado
inconsistente (o estado quebra alguma das regras supra-mencionadas, portanto).
Destas regras, a única cuja verificação merece alguma explicação é a da identificação
linhas/colunas repetidas.
Para garantir que tal nunca acontece, guardamos (em \texttt{Board}) a qualquer momento
dois \textit{sets}, cada um guardando \textit{strings} binárias que representam,
respetivamente, as linhas e colunas que já estão totalmente preenchidas no tabuleiro.
Esta solução com \textit{strings} binárias permite aumentar consideravelmente a
\textbf{eficiência} da verificação de igualdade entre linhas/colunas: é mais eficiente
comparar \textit{strings} que tuplos, por exemplo.

As jogadas podem ainda ser \textbf{possíveis} (corresponde apenas a não ser impossível)
ou \textbf{obrigatórias} (se forem possíveis e o seu conjugado não for).
Aqui, o conjugado de uma jogada corresponde à jogada que atua sobre a mesma célula,
mas com o valor \textbf{conjugado}: isto é, se uma jogada coloca 0 na posição (1,3),
a sua conjugada coloca 1 nessa mesma posição.

Em cada estado, verificamos sempre se há alguma célula vazia em que ambas as jogadas
sejam impossíveis.
Neste caso, qualquer jogada nessa célula levaria a um estado em que o tabuleiro é
inconsistente e não vale a pena prosseguir neste ramo da árvore de procura.

Sempre que haja ações obrigatórias por realizar, num dado estado, realizamo-las.
Isto é apenas lógico, visto que, por definição de obrigatoriedade, vamos ter de
as executar para chegar a qualquer tabuleiro solução (considerando o ramo
atual da árvore de procura, claro: uma jogada obrigatória num dado ramo poderá
não o ser na solução).
Assim, antecipando a sua execução, \textbf{reduzimos o número de nós da nossa árvore de procura}.

Sempre que tal não é possível, escolhemos um par de ações possíveis para qualquer
célula vazia do tabuleiro (note-se que uma vez que não há jogadas obrigatórias -
células em que apenas uma jogada é possível - nem células em que não seja possível jogar,
é necessariamente verdade que em qualquer célula vazia podemos colocar tanto um 0 como um 1).

Esta decisão de devolver sempre no máximo duas ações traduz-se em que o \textbf{\textit{branching factor}}
da nossa árvore seja 2.
Como vamos ver na análise experimental, isto é fundamental para a execução em tempo
eficiente da nossa solução.

Como a nossa procura é feita de forma a nunca alcançar estados inconsistentes, basta
que o nosso \texttt{goal\_test} verifique se está num estado completo - um estado
sem células vazias (nesse caso será necessariamente uma solução).

Tendo em conta a perspetiva dos CSP's (\textit{Constraint Satisfaction Problems})
abordada em aula, temos que na nossa solução:
\begin{itemize}
      \item A opção de devolver \textbf{no máximo duas opções}, ambas respetivas a apenas uma
            posição vazia, capitaliza na ideia de escolher uma variável de cada vez, visto
            que todas vão ter de ser escolhidas eventualmente.
            Como vimos em aula, isto pode reduzir o número de nós da árvore de procura
            em várias ordens de grandeza.
      \item A opção de devolver as jogadas obrigatórias sempre que possível é uma
            aplicação da heurística LCV (\textit{Least Constraining Value}).
            De facto, ao escolhermos um valor obrigatório para a variável não estamos
            a impor qualquer condição ao tabuleiro que ainda não estivesse imposta
            (mesmo que indiretamente).
\end{itemize}

\section*{Função Heurística}

\section*{Análise Experimental}

Tendo em conta a implementação descrita na página anterior, foram obtidos
os resultados experimentais (para os testes públicos fornecidos pela docência)
descritos na \textbf{Tabela 1}. Note-se que a coluna \texttt{Tempo de Execução (ms)}
corresponde à media de tempo de execução de cada teste, calculada recorrendo à ferramenta
\texttt{hyperfine} (com 250 execuções por teste, por procura). A heurística
utilizada para as procuras A* e Gananciosa é a referida na secção anterior.

Note-se que caso não se escolha apenas um par de ações possíveis por nível da árvore,
mas sim \textit{todas} as ações possíveis nesse nível, vamos ver uma alteração
drástica de desempenho para as procuras cegas. As procuras informadas conseguem, contudo,
manter o número de nós gerados e expandidos consistente com a implementação anterior.
Os resultados experimentais relativos a esta implementação encontram-se na \textbf{Tabela 2}.

% TODO: alterar o texto de cima, a alteração drástica foi só no teste 6 kek (e para todos)

Aqui, não faz particular sentido optar pela BFS (em detrimento da DFS): partindo
de uma configuração inicial, com $n$ posições vazias, é claro que uma solução
terá sempre de estar no nível $n$ da árvore de procura (não podendo estar mais
acima), visto que vamos sempre ter de executar $n$ jogadas para chegar a uma
solução. Assim sendo, os possíveis ganhos de uma procura em largura primeiro não são
aqui sentidos.

Adiciona-se ainda que, com um \textit{branching factor} tão pequeno (e com
um número de nós igualmente pequeno) as quantidades de nós expandidos e gerados
acabam por ser bastante próximas (e os tempos de execução obtidos também).

% hyperfine em cada procura com e sem branching factor a 2

% Assinalar que o branching factor tem bastante relevância na diferença entre
% procuras.

% TODO: falar de completude e consistência e cenas para cada procura

% TODO: conclusão, falar da que optámos no fim

\section*{Anexos}

% TODO: adicionar uma coluna com o tamanho do input

\begin{adjustbox}{width={\textwidth}, totalheight={\textheight}, keepaspectratio}
      \begin{tabular}{l cccc cccc cccc}
            \toprule
                  & \multicolumn{4}{c}{Tempo de Execução (ms)} & \multicolumn{4}{c}{Nós Gerados} & \multicolumn{4}{c}{Nós Expandidos}                                                                            \\
            \cmidrule(lr){2-5} \cmidrule(lr){6-9} \cmidrule(lr){10-13}
            Teste & BFS                                        & DFS                             & A*                                 & Gananciosa & BFS & DFS & A*  & Gananciosa & BFS & DFS & A*  & Gananciosa \\
            \midrule
            01    & 73.128                                     & 72.515                          & 73.633                             & 73.186     & 7   & 7   & 7   & 7          & 7   & 7   & 7   & 7          \\
            02    & 72.843                                     & 73.416                          & 73.337                             & 73.053     & 7   & 7   & 7   & 7          & 7   & 7   & 7   & 7          \\
            03    & 81.326                                     & 81.748                          & 82.538                             & 82.823     & 43  & 43  & 43  & 43         & 43  & 42  & 43  & 43         \\
            04    & 77.715                                     & 75.694                          & 77.806                             & 77.450     & 32  & 32  & 32  & 32         & 32  & 32  & 32  & 32         \\
            05    & 85.999                                     & 85.665                          & 86.816                             & 86.653     & 59  & 59  & 59  & 59         & 59  & 58  & 59  & 59         \\
            06    & 111.490                                    & 110.929                         & 113.014                            & 112.789    & 85  & 82  & 85  & 85         & 85  & 81  & 85  & 85         \\
            07    & 91.672                                     & 91.307                          & 92.653                             & 92.878     & 69  & 69  & 69  & 69         & 69  & 69  & 69  & 69         \\
            08    & 74.727                                     & 74.672                          & 75.366                             & 75.339     & 19  & 19  & 19  & 19         & 19  & 19  & 19  & 19         \\
            09    & 116.292                                    & 115.688                         & 118.930                            & 118.580    & 139 & 139 & 139 & 139        & 139 & 139 & 139 & 139        \\
            10    & 154.169                                    & 151.870                         & 156.289                            & 156.253    & 184 & 184 & 184 & 184        & 184 & 184 & 184 & 184        \\
            11    & 129.147                                    & 129.426                         & 133.098                            & 133.109    & 180 & 180 & 180 & 180        & 180 & 180 & 180 & 180        \\
            12    & 98.991                                     & 99.399                          & 102.951                            & 102.772    & 166 & 166 & 166 & 166        & 166 & 166 & 166 & 166        \\
            13    & 102.463                                    & 102.535                         & 107.096                            & 107.023    & 180 & 180 & 180 & 180        & 180 & 180 & 180 & 180        \\
            \bottomrule
      \end{tabular}
\end{adjustbox}

\vspace*{0.5cm}

\centering
Tabela 1: Resultados Experimentais, 1 variável por nível.

\vspace*{1cm}

\begin{adjustbox}{width={\textwidth}, totalheight={\textheight}, keepaspectratio}
      \begin{tabular}{l cccc cccc cccc}
            \toprule
                  & \multicolumn{4}{c}{Tempo de Execução (ms)} & \multicolumn{4}{c}{Nós Gerados} & \multicolumn{4}{c}{Nós Expandidos}                                                                              \\
            \cmidrule(lr){2-5} \cmidrule(lr){6-9} \cmidrule(lr){10-13}
            Teste & BFS                                        & DFS                             & A*                                 & Gananciosa & BFS & DFS  & A*  & Gananciosa & BFS & DFS  & A*  & Gananciosa \\
            \midrule
            01    & 72.970                                     & 73.094                          & 73.127                             & 73.297     & 7   & 7    & 7   & 7          & 7   & 7    & 7   & 7          \\
            02    & 73.474                                     & 73.014                          & 73.396                             & 73.446     & 7   & 7    & 7   & 7          & 7   & 7    & 7   & 7          \\
            03    & 82.725                                     & 81.771                          & 82.583                             & 82.932     & 60  & 49   & 51  & 51         & 57  & 42   & 44  & 44         \\
            04    & 76.408                                     & 76.588                          & 77.628                             & 77.514     & 32  & 32   & 32  & 32         & 32  & 32   & 32  & 32         \\
            05    & 95.799                                     & 86.219                          & 93.721                             & 92.753     & 209 & 80   & 149 & 140        & 189 & 57   & 114 & 104        \\
            06    & -                                          & 905.868                         & -                                  & -          & -   & 9537 & -   & -          & -   & 9430 & -   & -          \\
            07    & 92.043                                     & 92.060                          & 93.145                             & 92.655     & 69  & 69   & 69  & 69         & 69  & 69   & 69  & 69         \\
            08    & 74.632                                     & 74.724                          & 74.974                             & 75.032     & 19  & 19   & 19  & 19         & 19  & 19   & 19  & 19         \\
            09    & 115.589                                    & 116.284                         & 118.702                            & 118.603    & 139 & 139  & 139 & 139        & 139 & 139  & 139 & 139        \\
            10    & 152.918                                    & 151.920                         & 156.084                            & 156.218    & 184 & 184  & 184 & 184        & 184 & 184  & 184 & 184        \\
            11    & 128.591                                    & 129.511                         & 133.295                            & 133.087    & 180 & 180  & 180 & 180        & 180 & 180  & 180 & 180        \\
            12    & 98.737                                     & 99.028                          & 102.885                            & 102.364    & 166 & 166  & 166 & 166        & 166 & 166  & 166 & 166        \\
            13    & 102.898                                    & 102.658                         & 106.931                            & 107.385    & 180 & 180  & 180 & 180        & 180 & 180  & 180 & 180        \\
            \bottomrule
      \end{tabular}
\end{adjustbox}

\vspace*{0.5cm}

\centering
Tabela 2: Resultados Experimentais, todas as ações possíveis por nível.

\end{document}
